\documentclass[10pt]{article}
\begin{document}
\evensidemargin = 54pt
\topmargin = 18pt
\begin{center}
    {\bfseries\Huge PyFi Methodology}\\
\end{center}

\begin{flushleft}
    This document is intended to provide an explanation of the methodology used in the PyFi package.
    This will hopefully make review easier than trying to read code.
\end{flushleft}

\section{Basic Financial Formulae}
\subsection*{Present/Future Value of Cash Flows}
These calculations are performed in the \textbf{pv(.)} and \textbf{fv(.)} functions.\\
    \begin{equation}
        pv = \sum_{t=1}^n CF_t * (1 + (apr*dt))^{-t*dt}
    \end{equation}
and\\
    \begin{equation}
        fv = pv*(1 + (apr*dt))^{n}
    \end{equation}
where $CF_t$ is the list of cash flows, $apr$ is the annual nominal interest rate, 
$dt$ is the size of the time-step between cash flows, and $n$ is the number of periods. Note that 
$T/n = dt$, so

\end{document}
